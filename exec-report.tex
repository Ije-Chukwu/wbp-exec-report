% Options for packages loaded elsewhere
\PassOptionsToPackage{unicode}{hyperref}
\PassOptionsToPackage{hyphens}{url}
%
\documentclass[
]{article}
\usepackage{amsmath,amssymb}
\usepackage{iftex}
\ifPDFTeX
  \usepackage[T1]{fontenc}
  \usepackage[utf8]{inputenc}
  \usepackage{textcomp} % provide euro and other symbols
\else % if luatex or xetex
  \usepackage{unicode-math} % this also loads fontspec
  \defaultfontfeatures{Scale=MatchLowercase}
  \defaultfontfeatures[\rmfamily]{Ligatures=TeX,Scale=1}
\fi
\usepackage{lmodern}
\ifPDFTeX\else
  % xetex/luatex font selection
\fi
% Use upquote if available, for straight quotes in verbatim environments
\IfFileExists{upquote.sty}{\usepackage{upquote}}{}
\IfFileExists{microtype.sty}{% use microtype if available
  \usepackage[]{microtype}
  \UseMicrotypeSet[protrusion]{basicmath} % disable protrusion for tt fonts
}{}
\makeatletter
\@ifundefined{KOMAClassName}{% if non-KOMA class
  \IfFileExists{parskip.sty}{%
    \usepackage{parskip}
  }{% else
    \setlength{\parindent}{0pt}
    \setlength{\parskip}{6pt plus 2pt minus 1pt}}
}{% if KOMA class
  \KOMAoptions{parskip=half}}
\makeatother
\usepackage{xcolor}
\usepackage[margin=1in]{geometry}
\usepackage{color}
\usepackage{fancyvrb}
\newcommand{\VerbBar}{|}
\newcommand{\VERB}{\Verb[commandchars=\\\{\}]}
\DefineVerbatimEnvironment{Highlighting}{Verbatim}{commandchars=\\\{\}}
% Add ',fontsize=\small' for more characters per line
\usepackage{framed}
\definecolor{shadecolor}{RGB}{248,248,248}
\newenvironment{Shaded}{\begin{snugshade}}{\end{snugshade}}
\newcommand{\AlertTok}[1]{\textcolor[rgb]{0.94,0.16,0.16}{#1}}
\newcommand{\AnnotationTok}[1]{\textcolor[rgb]{0.56,0.35,0.01}{\textbf{\textit{#1}}}}
\newcommand{\AttributeTok}[1]{\textcolor[rgb]{0.13,0.29,0.53}{#1}}
\newcommand{\BaseNTok}[1]{\textcolor[rgb]{0.00,0.00,0.81}{#1}}
\newcommand{\BuiltInTok}[1]{#1}
\newcommand{\CharTok}[1]{\textcolor[rgb]{0.31,0.60,0.02}{#1}}
\newcommand{\CommentTok}[1]{\textcolor[rgb]{0.56,0.35,0.01}{\textit{#1}}}
\newcommand{\CommentVarTok}[1]{\textcolor[rgb]{0.56,0.35,0.01}{\textbf{\textit{#1}}}}
\newcommand{\ConstantTok}[1]{\textcolor[rgb]{0.56,0.35,0.01}{#1}}
\newcommand{\ControlFlowTok}[1]{\textcolor[rgb]{0.13,0.29,0.53}{\textbf{#1}}}
\newcommand{\DataTypeTok}[1]{\textcolor[rgb]{0.13,0.29,0.53}{#1}}
\newcommand{\DecValTok}[1]{\textcolor[rgb]{0.00,0.00,0.81}{#1}}
\newcommand{\DocumentationTok}[1]{\textcolor[rgb]{0.56,0.35,0.01}{\textbf{\textit{#1}}}}
\newcommand{\ErrorTok}[1]{\textcolor[rgb]{0.64,0.00,0.00}{\textbf{#1}}}
\newcommand{\ExtensionTok}[1]{#1}
\newcommand{\FloatTok}[1]{\textcolor[rgb]{0.00,0.00,0.81}{#1}}
\newcommand{\FunctionTok}[1]{\textcolor[rgb]{0.13,0.29,0.53}{\textbf{#1}}}
\newcommand{\ImportTok}[1]{#1}
\newcommand{\InformationTok}[1]{\textcolor[rgb]{0.56,0.35,0.01}{\textbf{\textit{#1}}}}
\newcommand{\KeywordTok}[1]{\textcolor[rgb]{0.13,0.29,0.53}{\textbf{#1}}}
\newcommand{\NormalTok}[1]{#1}
\newcommand{\OperatorTok}[1]{\textcolor[rgb]{0.81,0.36,0.00}{\textbf{#1}}}
\newcommand{\OtherTok}[1]{\textcolor[rgb]{0.56,0.35,0.01}{#1}}
\newcommand{\PreprocessorTok}[1]{\textcolor[rgb]{0.56,0.35,0.01}{\textit{#1}}}
\newcommand{\RegionMarkerTok}[1]{#1}
\newcommand{\SpecialCharTok}[1]{\textcolor[rgb]{0.81,0.36,0.00}{\textbf{#1}}}
\newcommand{\SpecialStringTok}[1]{\textcolor[rgb]{0.31,0.60,0.02}{#1}}
\newcommand{\StringTok}[1]{\textcolor[rgb]{0.31,0.60,0.02}{#1}}
\newcommand{\VariableTok}[1]{\textcolor[rgb]{0.00,0.00,0.00}{#1}}
\newcommand{\VerbatimStringTok}[1]{\textcolor[rgb]{0.31,0.60,0.02}{#1}}
\newcommand{\WarningTok}[1]{\textcolor[rgb]{0.56,0.35,0.01}{\textbf{\textit{#1}}}}
\usepackage{graphicx}
\makeatletter
\def\maxwidth{\ifdim\Gin@nat@width>\linewidth\linewidth\else\Gin@nat@width\fi}
\def\maxheight{\ifdim\Gin@nat@height>\textheight\textheight\else\Gin@nat@height\fi}
\makeatother
% Scale images if necessary, so that they will not overflow the page
% margins by default, and it is still possible to overwrite the defaults
% using explicit options in \includegraphics[width, height, ...]{}
\setkeys{Gin}{width=\maxwidth,height=\maxheight,keepaspectratio}
% Set default figure placement to htbp
\makeatletter
\def\fps@figure{htbp}
\makeatother
\setlength{\emergencystretch}{3em} % prevent overfull lines
\providecommand{\tightlist}{%
  \setlength{\itemsep}{0pt}\setlength{\parskip}{0pt}}
\setcounter{secnumdepth}{-\maxdimen} % remove section numbering
\ifLuaTeX
  \usepackage{selnolig}  % disable illegal ligatures
\fi
\usepackage{bookmark}
\IfFileExists{xurl.sty}{\usepackage{xurl}}{} % add URL line breaks if available
\urlstyle{same}
\hypersetup{
  pdftitle={EXECUTIVE REPORT (PART A): PROFESSIONAL WORK BASED PLACEMENT IN HEALTH DATA SCIENCE SCIENCE},
  hidelinks,
  pdfcreator={LaTeX via pandoc}}

\title{EXECUTIVE REPORT (PART A): PROFESSIONAL WORK BASED PLACEMENT IN
HEALTH DATA SCIENCE SCIENCE}
\author{}
\date{\vspace{-2.5em}}

\begin{document}
\maketitle

\begin{center}\rule{0.5\linewidth}{0.5pt}\end{center}

\subsection{Project Topic: DHS Data Management and Analysis of Gender
Inequality in Reproductive Women across LMICs using IPUMS-DHS
Dataset}\label{project-topic-dhs-data-management-and-analysis-of-gender-inequality-in-reproductive-women-across-lmics-using-ipums-dhs-dataset}

\subsubsection{Project Background}\label{project-background}

The Biostatistics and Health Data Science Group, is a multi-disciplinary
academic research and teaching under the IAHS characteristic by
collaborative research, consultancy and training across clinical,
biological and global health domains. In the global health domain where
I was assigned to, the data used to conduct the research as well as for
training purposes are collected from a number of secure sources,
including the \href{https://dhsprogram.com}{The DHS-Program}.

The DHS-Program, funded by USAID collects nationally representative
global health data, to monitor and evaluate population, health, and
nutrition programs, providing data to track approximately 30 SDG
indicators. They provides these data for tracking as well as measure to
track them, contributing significantly towards achieving the SDG 3 and 5
(The DHS Program, 2025).

However, the DHS-Program has been suspended and currently undergoing
review for further funding. During the period of this review, new
registrations are not being accepted, hence restricting access to
datasets commonly used by undergraduate and post graduate students for
their theses and training, especially in LMICs, thereby significantly
hampering preparations for future national and global health leadership
training in addition to other far-reaching effects.

My project focused on collecting, organizing, merging and analyzing,
datasets from DHS-program relevant to our global health projects. While
this mitigates the recent suspension of the DHS-program for students and
researchers within the team working on global health projects, it also
gave me an opportunity to familiarize with global health data and
perform exploratory data analysis on aspects of Gender Inequality
including Female Genital Mutilation, Intimate Partner Violence and
Autonomy of Health Care Decision Making which are often intertwined and
are prevalent issues for women of child bearing age in LMICs(Wessells \&
Kostelny, 2022).

\subsubsection{Project Aim}\label{project-aim}

This project achieved two aims

\begin{enumerate}
\def\labelenumi{\arabic{enumi}.}
\tightlist
\item
  Created a global health data repository of DHS Datasets for 38 years
  (1984-2022)
\item
  Pooled Cross Country Exploratory Data Analysis of Gender Inequalities
  in women of child bearing age.
\end{enumerate}

\subsubsection{Methods}\label{methods}

The project was carried out in three phases and documentation was
ensured for transparency and reproducibility of the workflow and
analysis results.

The data was from DHS-Program website using my supervisor's login. To
access datasets, new users must
\href{https://dhsprogram.com/data/new-user-registration.cfm}{register
for an account} on the \href{https://dhsprogram.com}{The DHS-Program}
website.

\paragraph{Phase 1: Auto-download of DHS
Datasets}\label{phase-1-auto-download-of-dhs-datasets}

A structured reproducible workflow was scripted using R Markdown which
serves as a comprehensive toolkit for accessing, processing, and locally
managing DHS downloads, enabling seamless data retrieval for
collaborative research in support of global health studies. It ensure
secure data access, automates downloads, and systematically unzips,
organizes and saves the datasets in hierarchical file
structure.\texttt{FileName/CountryName/SurveyYear/DataType}. The
workflow is specifically for DHS Datasets in SPSS and STATA formats as
specified in my project tasks.

\paragraph{Phase 2: DHS IR File Merge (Pilot
merge)}\label{phase-2-dhs-ir-file-merge-pilot-merge}

A structured, reproducible workflow was developed to merge DHS
Individual Recode (IR) datasets for 2 countries (Kenya and Tanzania
2022) using SPSS Syntax. A cross-Country unique identifiers
\texttt{UCASEID} was created by concatenating Country-cluster and case
IDs. Subsets containing the \texttt{UCASEID} and relevant IPV variables
were saved and merged using SPSS commands. This workflow can be adapted
for additional countries and survey rounds, and replicated for different
variables, provided that the variable names, labels, and meanings are
first confirmed to be consistent according to the DHS Recode Manual (The
DHS Program, 2025). See syntax of workflow in
\href{Appendix\%201}{Appendix 1}.

\paragraph{Phase 3: Exploratory Data Analysis and Results using
IPUMS-DHS
Data}\label{phase-3-exploratory-data-analysis-and-results-using-ipums-dhs-data}

Exploratory data analysis was done using datasets from
\href{https://www.idhsdata.org/idhs/}{IPUMS-DHS website} which are
harmonized DHS survey datasets across countries and over time. Initial
Cross-tabulations was done for key variables using SPSS, SPSS output
tables were cleaned excel with steps documented (see
\href{Appendix\%202}{Appendix 2} ) and results imported into R to create
visual interactive plots.

\subparagraph{I. IPV: Percentage of Women Slapped in Last 12 month
(frequency) by an Intimate Partner, variable code=
(DVPSLAPFQ)}\label{i.-ipv-percentage-of-women-slapped-in-last-12-month-frequency-by-an-intimate-partner-variable-code-dvpslapfq}

This plot presents the distribution of women's reported experiences with
intimate partner violence (IPV) across countries. The response
categories include: ``Not ever slapped,'' ``Often during last 12
months,'' ``Sometimes during last 12 months,'' ``Not at all in last 12
months,'' and ``Yes, timing and frequency unknown.'' Most countries show
that a significant proportion of women have never been slapped by an
intimate partner, but in many settings, notable percentages report being
slapped at least sometimes or often within the past year. Variability
across countries is visible, with some (e.g., Sao Tome and Principe,
Zimbabwe) having higher frequencies of violence, and others (e.g.,
India, Senegal) showing larger shares of respondents reporting no
experience of IPV.

\subparagraph{II. FGM: percentage of ever circumcised women within
Country, variable code=
(FCCIRC)}\label{ii.-fgm-percentage-of-ever-circumcised-women-within-country-variable-code-fccirc}

This plot presents the percentage response of women who have experienced
female genital mutilation/cutting (FGM/C) within Country, with responses
categorized as ``yes,'' ``no,'' and ``don't know.'' There is wide
Country variation: nations like Guinea, Sierra Leone, Mali, Gambia, and
Egypt show extremely high percentages of women reporting being
circumcised (often over 80\%), while countries such as Ghana, Cameroon,
Tanzania, and others report relatively low response rate. The ``don't
know'' response is almost negligible in most contexts. The significant
Country-to-Country differences reflects varying cultural, legal, and
historical norms about FGM/C practices.

\subparagraph{III. AHCDM: percentage of women who have the final say on
their health care within Country, variable code=
DECFEMHCARE}\label{iii.-ahcdm-percentage-of-women-who-have-the-final-say-on-their-health-care-within-country-variable-code-decfemhcare}

The chart shows women's reported autonomy and roles in health care
decision-making by \texttt{Country} and response categories
\texttt{response\_ahcdm}. In many countries, the largest proportion of
women say decisions are made ``with their husband/partner'' or by their
``husband/partner'' alone, reflecting persistent gender norms around
health autonomy. However, countries such as Mozambique, Lesotho, and
Madagascar display higher shares for ``Woman alone,'' indicating
stronger female decision-making autonomy. ``Woman and someone else'' and
``Family elders/relatives'' are minor categories in most contexts,
suggesting these are less common arrangements for household health
decisions.

\paragraph{Output}\label{output}

Output files including dataset, analysis and results are saved to One
drive folder in the below order

\begin{verbatim}
DHS-Download Task
├── [DHS_Downloads]
└── [Downloads report, metadata, log]

[Gender Inequalities]
├── [DHS]
│   ├── [dhs-ir-piolt-merge-KE8_TZ8]
│   └── [planning-and-var-map]
└── [IPUMS]
    ├── [ipums-analysis]
    │   ├── [r-project-files-exec-report]
    │   └── [spss-analysis]
    ├── [ipums-data-extracts-comd-files]
    ├── [ipums-ir-dataset]
    └── [ipums-planning-and-var-map]

    
\end{verbatim}

\subsubsection{Implications for the
Organisation:}\label{implications-for-the-organisation}

\begin{itemize}
\tightlist
\item
  Easy access to DHS Global Datasets.
\item
  Data Preservation pending the resumption of the DHS program and
  mitigation against possible future program suspension.
\item
  Continuity of trainings, including future and ongoing training and
  projects by students and researchers working on global health
  projects.
\end{itemize}

\subsubsection{References}\label{references}

The DHS Program. (2025).\emph{Sustainable Development Goals.}
\url{https://dhsprogram.com/topics/sdgs/index.cfm} (Accessed August 28,
2025)

The DHS Program. (2025). \emph{Merging datasets.}
\url{https://dhsprogram.com/data/Merging-datasets.cfm} (Accessed
September 1, 2025)

Wessells, M. G., \& Kostelny, K. (2022). The psychosocial impacts of
intimate partner violence against women in LMIC contexts: Toward a
holistic approach. \emph{International Journal of Environmental Research
and Public Health, 19(21), 14488.}
\url{https://doi.org/10.3390/ijerph192114488}*

\subsubsection{Appendix 1}\label{appendix-1}

\begin{Shaded}
\begin{Highlighting}[]
\SpecialCharTok{*}\NormalTok{SPSS}
\SpecialCharTok{*}\NormalTok{ Encoding}\SpecialCharTok{:}\NormalTok{ UTF}\FloatTok{{-}8.}
\SpecialCharTok{*}\NormalTok{SPSS Version }\DecValTok{30}\NormalTok{.}\DecValTok{0}\NormalTok{.}\FloatTok{0.0}\NormalTok{(}\DecValTok{172}\NormalTok{)}
\SpecialCharTok{*}\NormalTok{  Encoding}\SpecialCharTok{:}\NormalTok{ UTF}\FloatTok{{-}8.}


\SpecialCharTok{*}\NormalTok{Check Recode file to confirm  variable names context match. For this pilot merging, KEIR8CFL.SAV and TZIR82FL.SAV were conducted }\ControlFlowTok{in}\NormalTok{ the same year and survey }\FunctionTok{phase}\NormalTok{ (Ist Survey conducted }\ControlFlowTok{in}\NormalTok{ DHS Phase }\DecValTok{8}\NormalTok{, }\ControlFlowTok{in} \DecValTok{2022}\NormalTok{).}

\SpecialCharTok{*}\NormalTok{KEIR8CFL.SAV however is a continuous DHS Dataset. Create a copy of original dataset as these changes will over}\SpecialCharTok{{-}}\NormalTok{write the original dataset. UNless otherwise specified as }\ControlFlowTok{in}\NormalTok{ Step }\DecValTok{2}

\SpecialCharTok{*}\NormalTok{STep1}\SpecialCharTok{:}\NormalTok{ Create Unique ID using V000 and Case ID variables from both files. to merge from Dataset }\DecValTok{1}\NormalTok{( KEIR8CFL.SAV )}

\SpecialCharTok{*}\NormalTok{Unique ID }\ControlFlowTok{for}\NormalTok{ Kenya; Dataset }\DecValTok{1}\NormalTok{( KEIR8CFL.SAV ).}

\NormalTok{DATASET ACTIVATE DataSet1.}
\NormalTok{STRING  }\FunctionTok{UCASEID}\NormalTok{ (A20).}
\NormalTok{COMPUTE UCASEID}\OtherTok{=}\FunctionTok{CONCAT}\NormalTok{(V000,CASEID).}
\NormalTok{VARIABLE LABELS  UCASEID }\StringTok{\textquotesingle{}Unique Case ID\textquotesingle{}}\NormalTok{.}
\NormalTok{EXECUTE.}


\SpecialCharTok{*}\NormalTok{Unique ID }\ControlFlowTok{for}\NormalTok{ Tanzania; Dataset }\DecValTok{2}\NormalTok{( TZIR82FL.SAV ).}

\NormalTok{DATASET ACTIVATE DataSet2.}
\NormalTok{STRING  }\FunctionTok{UCASEID}\NormalTok{ (A20).}
\NormalTok{COMPUTE UCASEID}\OtherTok{=}\FunctionTok{CONCAT}\NormalTok{(V000,CASEID).}
\NormalTok{VARIABLE LABELS  UCASEID }\StringTok{\textquotesingle{}Unique Case ID\textquotesingle{}}\NormalTok{.}
\NormalTok{EXECUTE.}



\SpecialCharTok{*}\NormalTok{Step }\DecValTok{2}\SpecialCharTok{:}\NormalTok{ Select Unique case ID along with IPV variables from both datasets }\ControlFlowTok{for}\NormalTok{ merging. Save them with a different name. Modify file path.}

\NormalTok{DATASET ACTIVATE DataSet1.}
\NormalTok{SAVE OUTFILE}\OtherTok{=}\StringTok{\textquotesingle{}C:\textbackslash{}Users\textbackslash{}Desktop\textbackslash{}\_KEIR8CFL.SAV\textquotesingle{}}
  \SpecialCharTok{/}\NormalTok{KEEP UCASEID V000 V001 V003 V004 V005 V006 V007 G100 G101 G102 G103 G104 G105 G107 V005.}

\NormalTok{DATASET ACTIVATE DataSet2.}
\NormalTok{SAVE OUTFILE}\OtherTok{=}\StringTok{\textquotesingle{}C:\textbackslash{}Users\textbackslash{}Desktop\textbackslash{}\_TZIR82FL.SAV\textquotesingle{}}
  \SpecialCharTok{/}\NormalTok{KEEP UCASEID V000 V001 V003 V004 V005 V006 V007 G100 G101 G102 G103 G104 G105 G107 V005.}

\SpecialCharTok{*}\NormalTok{Open \_KEIR8CFL.SAV and \_TZIR82FL.SAV as Datasets }\DecValTok{3}\NormalTok{ and }\DecValTok{4}\NormalTok{ respectively}


\SpecialCharTok{*}\NormalTok{Step }\DecValTok{3}\SpecialCharTok{:}\NormalTok{ Merge all variables.}
\NormalTok{DATASET ACTIVATE DataSet3.}
\NormalTok{ADD FILES }\SpecialCharTok{/}\NormalTok{FILE}\OtherTok{=}\ErrorTok{*}
  \ErrorTok{/}\NormalTok{FILE}\OtherTok{=}\StringTok{\textquotesingle{}DataSet4\textquotesingle{}}\NormalTok{.}
\NormalTok{EXECUTE.}

\SpecialCharTok{*}\NormalTok{By default, the active }\FunctionTok{dataset}\NormalTok{ (Dataset3 \_KEIR8CFL.SAV) is modified to contain the merged cases from the other }\FunctionTok{dataset}\NormalTok{ (Dataset4  \_TZIR82FL.SAV).}


\NormalTok{SAVE OUTFILE}\OtherTok{=}\StringTok{\textquotesingle{}C:\textbackslash{}Users\textbackslash{}Desktop\textbackslash{}KE8{-}TZ8{-}ir{-}ipv.SAV\textquotesingle{}}
  \SpecialCharTok{/}\NormalTok{COMPRESSED.}
\end{Highlighting}
\end{Shaded}

\subsubsection{Appendix 2}\label{appendix-2}

\begin{Shaded}
\begin{Highlighting}[]
\SpecialCharTok{*}\NormalTok{ Encoding}\SpecialCharTok{:}\NormalTok{ UTF}\DecValTok{{-}8}
\SpecialCharTok{*}\NormalTok{Version }\DecValTok{29}\NormalTok{.}\DecValTok{0}\NormalTok{.}\FloatTok{2.0}\NormalTok{ (}\DecValTok{20}\NormalTok{)}
\NormalTok{Naming conventions }\ControlFlowTok{for}\NormalTok{ CROSS TABULATIONS results }\ControlFlowTok{for}\NormalTok{ further analysis}
\FloatTok{1.}\NormalTok{ ipv}\SpecialCharTok{:}\NormalTok{ percentage of women slapped }\ControlFlowTok{in}\NormalTok{ last }\DecValTok{12} \FunctionTok{month}\NormalTok{ (frequency), variable code}\OtherTok{=}\NormalTok{ (DVPSLAPFQ)}
\FloatTok{2.}\NormalTok{ fgm}\SpecialCharTok{:}\NormalTok{ percentage of ever circumcised women within Country, variable code}\OtherTok{=}\NormalTok{ (FCCIRC)}
\FloatTok{3.}\NormalTok{ ahcdm}\SpecialCharTok{:}\NormalTok{ percentage of women who have the final say on their health care within Country, variable code}\OtherTok{=}\NormalTok{ (FCCIRC)}


\SpecialCharTok{*}\NormalTok{Load datset.}
\NormalTok{ GET}
\NormalTok{  FILE}\OtherTok{=}\StringTok{\textquotesingle{}C:\textbackslash{}Users\textbackslash{}Desktop\textbackslash{}ipums{-}ir{-}dataset.sav\textquotesingle{}}\NormalTok{.}



\NormalTok{DATASET ACTIVATE DataSet1.}

\NormalTok{CROSSTABS}
  \SpecialCharTok{/}\NormalTok{TABLES}\OtherTok{=}\NormalTok{Country BY DVPSLAPFQ}
  \SpecialCharTok{/}\NormalTok{FORMAT}\OtherTok{=}\NormalTok{AVALUE TABLES}
  \SpecialCharTok{/}\NormalTok{CELLS}\OtherTok{=}\NormalTok{COUNT ROW COLUMN }
  \SpecialCharTok{/}\NormalTok{COUNT ROUND CELL.}


\NormalTok{CROSSTABS}
  \SpecialCharTok{/}\NormalTok{TABLES}\OtherTok{=}\NormalTok{ Country BY FCCIRC}
  \SpecialCharTok{/}\NormalTok{FORMAT}\OtherTok{=}\NormalTok{AVALUE TABLES}
  \SpecialCharTok{/}\NormalTok{CELLS}\OtherTok{=}\NormalTok{COLUMN }
  \SpecialCharTok{/}\NormalTok{COUNT ROUND CELL.}


\NormalTok{CROSSTABS}
  \SpecialCharTok{/}\NormalTok{TABLES}\OtherTok{=}\NormalTok{Country BY DECFEMHCARE}
  \SpecialCharTok{/}\NormalTok{FORMAT}\OtherTok{=}\NormalTok{AVALUE TABLES}
  \SpecialCharTok{/}\NormalTok{CELLS}\OtherTok{=}\NormalTok{COUNT ROW COLUMN }
  \SpecialCharTok{/}\NormalTok{COUNT ROUND CELL..}



\SpecialCharTok{*{-}{-}{-}{-}{-}{-}{-}{-}{-}{-}{-}{-}{-}{-}{-}{-}{-}{-}{-}{-}{-}{-}{-}{-}{-}{-}{-}{-}{-}{-}{-}{-}{-}{-}{-}{-}{-}{-}{-}{-}{-}{-}{-}{-}{-}{-}{-}{-}{-}{-}{-}{-}{-}{-}{-}{-}{-}{-}{-}{-}{-}{-}{-}{-}{-}{-}{-}{-}{-}{-}{-}}\NormalTok{.}


\SpecialCharTok{*}\NormalTok{For data cleaning }\ControlFlowTok{in}\NormalTok{ excel}
 \FloatTok{1.}\NormalTok{ remove}\SpecialCharTok{:}
\NormalTok{     First }\DecValTok{3}\NormalTok{ row headings}
 \FloatTok{2.}\NormalTok{ Name col1}\SpecialCharTok{:}\NormalTok{ Country}
 \FloatTok{3.}\NormalTok{ Populate Country column}
 \FloatTok{4.}\NormalTok{ Filter and remove}\SpecialCharTok{:}
    \SpecialCharTok{{-}}\NormalTok{ All row}\SpecialCharTok{/}\NormalTok{col Totals}
    \SpecialCharTok{{-}}\NormalTok{ cols}\SpecialCharTok{:}
\NormalTok{            Not }\ControlFlowTok{in}\NormalTok{ Universe col}
\NormalTok{            Missing }
    \SpecialCharTok{{-}}\NormalTok{rows}\SpecialCharTok{:} \ControlFlowTok{in}\NormalTok{ count}\SpecialCharTok{/}\NormalTok{\% col}
\NormalTok{             Blank}
\NormalTok{             All rows except \% within }\FunctionTok{Country}\NormalTok{ (}\ControlFlowTok{for}\NormalTok{ fgm and ahcdm)}
 \FloatTok{5.}\NormalTok{Number format is Percentage}
\end{Highlighting}
\end{Shaded}

\subsubsection{Appendix 3}\label{appendix-3}

\emph{Abbreviations} - DHS - IPUMS - USAID - LMICs - All IPUMS Variable
abbreviations available on
\href{https://www.idhsdata.org/idhs-action/variables/group}{IPUMS-DHS}.

\end{document}
